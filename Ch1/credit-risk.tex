\section{Tổng quan về rủi ro tín dụng và vấn đề quản trị rủi ro trong ngân hàng}
\subsection{Một số khái niệm}
\subsubsection{Khái niệm rủi ro tín dụng}
\paragraph{Rủi ro}
% Nhiều quan niệm rủi ro 
Có nhiều cách quan niệm khác nhau về rủi ro phụ thuộc vào lĩnh vực mà khái niệm này áp dụng.  Tuy nhiên, các quan niệm khác nhau nhìn chung đều coi rủi ro là các biến cố không mong đợi, gây ra thiệt hại và có thể đo lường được.

% Quan niệm trong NHTM
Trong các ngân hàng thương mại, rủi ro được hiểu là những biến cố có thể gây ra thiệt hại cho lợi nhuận hay thậm chí là nguy cơ phá sản của các ngân hàng.

% Tầm quan trọng
Trong hoạt động kinh tế nói chung và trong hoạt động ngân hàng nói riêng thì vấn đề rủi ro là không thể tránh khỏi. Vì thế, các ngân hàng  không thể loại bỏ được rủi ro mà chỉ có thể phát hiện kịp thời để có những biện pháp chủ động xử lý.

\paragraph{Rủi ro tín dụng}
Là rủi ro do một khách hàng hay một nhóm khách hàng vay vốn không trả được nợ cho ngân hàng. Trong kinh doanh ngân hàng, rủi ro tín dụng là loại rủi ro lớn nhất, thường xuyên xảy ra và gây hậu quả nặng nề, có khi dẫn đến phá sản cho ngân hàng.

Ngày nay, nhu cầu về vốn để mở rộng sản xuất kinh doanh, cải tiến trang thiết bị kỹ thuật, nâng cao công nghệ và các nhu cầu phục vụ sản xuất kinh doanh luôn tăng lên. Để đáp ứng nhu cầu này, các NHTM cũng phải luôn mở rộng quy mô hoạt động tín dụng, điều đó có nghĩa là rủi ro tín dụng cũng phát sinh nhiều hơn.

Rủi ro tín dụng là loại rủi ro phức tạp nhất, việc quản lý và phòng ngừa nó rất khó khăn, nó có thể xảy ra ở bất cứ đâu, bất cứ lúc nào... Rủi ro tín dụng nếu không được phát hiện và sử lý kịp thời sẽ nảy sinh các rủi ro khác.

Rủi ro tín dụng có thể rơi vào một trong các loại sau:
\begin{itemize}
\item Rủi ro vỡ nợ: Rủi ro của các khoản thiệt hại phát sinh từ người đi vay không có khả năng thanh toán đầy đủ khoản nợ hoặc đã quá thời hạn quy định mà không có khả năng trả nợ. Rủi ro vỡ nợ có thể tác động đến tất cả các giao dịch nhạy cảm với tín dụng bao gồm các khoản vay, chứng khoán và các công cụ phái sinh.
\item Rủi ro tập trung: Rủi ro này xuất hiện khi ngân hàng thương mại cho vay một khách hàng hoặc một nhóm khách hàng có liên quan với mức tín dụng quá lớn so với năng lực tài chính của NHTM tại một thời điểm nào đó. Khi khách hàng gặp rủi ro trong hoạt động và không thể thanh toán nợ đúng hạn, NHTM cho vay có thể gặp vấn đề thanh khoản.
\item Rủi ro quốc gia: Rủi ro khi những thay đổi kinh tế hay chính trị tại nước ngoài, ví dụ như thiếu dự trữ tiền tệ (hối đoái), sẽ gây chậm trễ thanh toán tiền vay cho các  ngân hàng tín dụng, cơ quan kiểm soát ngoại hối hoặc gây mất khoản nợ. Rủi ro thuộc về quốc gia có phạm vi rộng hơn rủi ro chủ quyền, vì nó xem xét lãi suất hoàn trả nợ từ nhũng người vay tư nhân cũng như chính phủ trung ương. Các ngân hàng dành riêng các quỹ trong một tài khoản dự trữ, gọi là dự trữ rủi ro chuyển giao được phân bố, làm khoản đệm đối phó với những khoản lỗi nợ khó đòi có thể xảy ra từ các khoản vay nước ngoài.
\end{itemize}

\subsection{Một số nguyên nhân của rủi ro tín dụng}
Thực tế kinh doanh của Ngân hàng trong thời gian qua cho thấy rủi ro tín dụng xảy ra là do những nguyên nhân sau:

\paragraph{Nguyên nhân từ phía Ngân hàng}
\begin{itemize}
\item Ngay hàng đưa ra chính sách tín dụng không phù hợp với nền kinh tế và thể lệ cho vay còn sơ hở để khách hàng lợi dụng chiếm đoạt vốn của Ngân hàng.
\item Do cán bộ Ngân hàng chưa chấp hành đúng quy trình cho vay như: không đánh giá đầy đủ chính xác khách hàng trước khi cho vay, cho vay khống, thiếu tài sản đảm bảo, cho vay vượt tỷ lệ an toàn. Đồng thời cán bộ Ngân hàng không kiểm tra, giám sát chặt chẽ về tình hình sử dụng vốn vay của khách hàng.
\item Do trình độ nghiệp vụ của cán bộ tín dụng còn nên việc đánh giá các dự án, hồ sơ xin vay còn chưa tốt, còn xảy ra tình trạng dự án thiếu tính khả thi mà vẫn cho vay.
\item Cán bộ Ngân hàng còn thiếu tinh thần trách nhiệm, vi phạm đạo đức kinh doanh như: thông đồng với khách hàng lập hồ sơ giả để vay vốn, xâm tiêu khi giải ngân hay thu nợ, đôi khi còn nể nang trong quan hệ khách hàng.
\item Ngân hàng đôi khi quá chú trọng về lợi nhuận, đặt những khoản vay có lợi nhuân cao hơn những khoản vay lành mạnh.
\item Do áp lực cạnh tranh với các Ngân hàng khác.
\item Do tình trạng tham nhũng, tiêu cực diễn ra trong nội bộ Ngân hàng
\end{itemize}

\paragraph{Nguyên nhân từ phía khách hàng.}

\begin{itemize}
\item Người vay vốn sử dụng vốn vay sai mục đích, sử dụng vào các hoạt động có rủi ro cao dẫn đến thua lỗ không trả được nợ cho Ngân hàng.
\item Do trình độ kinh doanh yếu kếm, khả năng tổ chức điều hành sản xuất kinh doanh của lãnh đạo còn hạn chế.
\item Doanh nghiệp vay ngắn hạn để đầu tư vào tài sản lưu động và cố định.
\item Doanh nghiệp sản xuất kinh doanh thiếu sự linh hoạt, không cải tiến quy trình công nghệ, không trang bị máy móc hiện đại, không thay đổi mẫu mã hoặc nghiên cứu nâng cao chất lượng sản phẩm...dẫn tới sản phẩm sản xuất ra thiếu sự cạnh tranh, bị ứ đọng trên thị trường khiến cho doanh nghiệp không có khả năng thu hồi vốn trả nợ cho Ngân hàng.
\item Do bản thân doanh nghiệp có chủ ý lừa gạt, chiếm dụng vốn của Ngân hàng, dùng một loại tài sản thế chấp đi vay nhiều nơi, không đủ năng lực pháp nhân.
\end{itemize}
  
\paragraph{Nguyên nhân khác.}

\begin{itemize}
\item Do sự thay đổi bất thường của các chính sách, do thiên tai bão lũ, do nền kinh tế không ổn định.... khiến cho cả Ngân hàng và khách hàng không thể ứng phó kịp.
\item Do môi trường pháp lý lỏng lẻo, thiếu đồng bộ, còn nhiều sơ hở dẫn tới không kiểm soát được các hiện tượng lừa đảo trong việc sử dụng vốn của khách hàng.
\item Do sự biến động về chính trị - xã hội trong và ngoài nước gây khó khăn cho doanh nghiệp dẫn tới rủi ro cho Ngân hàng.
\item Ngân hàng không theo kịp đà phát triển của xã hội, nhất là sự bất cập trong trình độ chuyên môn cũng như công nghệ Ngân hàng.
\item Do sự biến động của kinh tế như suy thoái kinh tế, biến động tỷ giá, lạm phát gia tăng ảnh hưởng tới doanh nghiệp cũng như Ngân hàng.
\item Sự bất bình đẳng trong đối sử của Nhà nước dành cho các NHTM khác nhau.
\item Chính sách Nhà nước chậm thay đổi hoặc chưa phù hợp với tình hình phát triển của đất nước.
\end{itemize}

\subsection{Sự cần thiết phải phòng ngừa rủi ro tín dụng.}

\paragraph{Đối với bản thân Ngân hàng}
Các nhà kinh tế thường gọi Ngân hàng là “ngành kinh doanh rủi ro”. Thực tế đã chứng minh không một ngành nào mà khả năng dẫn đến rủi ro lại lớn như trong lĩnh vực kinh doanh tiền tệ- tín dụng. Ngân hàng phải gánh chịu những rủi ro không những do nguyên nhân chủ quan của mình, mà còn phải gánh chịu những rủi ro khách hàng gây ra. Vì vậy “rủi ro tín dụng của Ngân hàng không những là cấp số cộng mà có thể là cấp số nhân rủi ro của nền kinh tế”.

Khi rủi ro xảy ra, trước tiên lợi nhuận kinh doanh của Ngân hàng sẽ bị ảnh hưởng. Nếu rủi ro xảy ra ở mức độ nhỏ thì Ngân hàng có thể bù đắp bằng khoản dự phòng rủi ro (ghi vào chi phí) và bằng vốn tự có, tuy nhiên nó sẽ ảnh hưởng trực tiếp tới khả năng mở rộng kinh doanh của Ngân hàng. Nghiêm trọng hơn, nếu rủi ro xảy ra ở mức độ lớn, nguồn vốn của Ngân hàng không đủ bù đắp, vốn khả dụng bị thiếu, lòng tin của khách hàng giảm tất nhiên sẽ dẫn tới phá sản Ngân hàng. Vì vậy việc phòng ngừa và hạn chế rủi ro tín dụng là một việc làm cần thiết đối với các NHTM.

\paragraph{Đối với nền kinh tế}
Trong nền kinh tế thị trường, hoạt động kinh doanh của Ngân hàng liên quan đến rất nhiều các thành phần kinh tế từ cá nhân, hộ gia đình, các tổ chức kinh tế cho tới các tổ chức tín dụng khác. Vì vậy, kết quả kinh doanh của Ngân hàng phản ánh kết quả sản xuất kinh doanh của nền kinh tế và đương nhiên nó phụ thuộc rất lớn vào tình hình tổ chức sản xuất kinh doanh của các doanh nghiệp và khách hàng. Hoạt động kinh doanh của Ngân hàng không thể có kết quả tốt khi hoạt động kinh doanh của nền kinh tế chưa tốt hay nói cách khác hoạt động kinh doanh của Ngân hàng sẽ có nhiều rủi ro khi hoạt động kinh tế có nhiều rủi ro. Rủi ro xảy ra dẫn tới tình trạng mất ổn định trên thị trường tiền tệ, gây khó khăn cho các doanh nghiệp sản xuất kinh doanh, làm ảnh hưởng tiêu cực đối với mnền kinh tế và đời sống xã hội. Do đó, phòng ngừa và hạn chế rủi ro tín dụng không những là vấn đề sống còn với Ngân hàng mà còn là yêu cầu cấp thiết của nền kinh tế góp phần vào sự ổn định và phát triển của toàn xã hội.



\subsection{Tổng quan về chấm điểm tín dụng tại các ngân hàng}
\subsubsection{Khái niệm về chấm điểm tín dụng}
Chấm điểm tín dụng là một phương thức đế đánh giá rủi ro của những đổi tượng đi vay. 
Theo đó, ngân hàng sử dụng phương pháp thống kê, nghiên cứu dữ liệu đế đánh giá rủi ro của việc cho người vay. 
Phương pháp này đưa ra “điểm” mà ngân hàng có thế sử dụng đế xếp loại những người xin vay xét về độ mạo hiểm. Để tạo dựng một hình mẫu chấm điểm, hay một “bảng điểm”, thì những người nghiên cứu phân tích số liệu trong quá khứ về các khoản vay trước đó đế quyết định những đặc điểm của những người đi vay nào là hữu ích trong việc phỏng đoán xem liệu khoản vay đó có phát huy tốt tác dụng không.

Một mô hình được thiết kế tốt sẽ đưa ra tỷ lệ điểm cao nhiều hơn cho những người đi vay có khả năng sử dụng vốn vay hiệu quả và ngược lại, tỷ lệ phần trăm điểm thấp nhiều hơn cho những người đi vay mà những khoản vay ít phát huy tác dụng. Nhưng không có mô hình nào là hoàn hảo, cho nên đôi khi có những khách hàng có khả năng không trả được nợ lại nhận được điểm cao hơn.
Thông tin của những người đi vay được thu nhận từ đơn đăng ký của đối tượng cho vay như: thu nhập hàng tháng của cá nhân/doanh nghiệp đi vay, khoản nợ đọng, tài sản tài chính, khoản thời gian mà doanh nghiệp hoạt động trong lĩnh vực kinh doanh của mình, liệu doanh nghiệp đã tùng phạm lỗi trong một khoản vay trước đó hay không, loại tài khoản ngân hàng mà doanh nghiệp đi vay có. Tất cả đều là những yếu tố tiềm năng có khả năng đánh giá được khoản vay mà sẽ được xem xét để sử dụng trong bảng điểm.

Phân tích tổng hợp liên quan đến khoản vay từ những biến số ở trên được sử dụng để tìm ra sự kết hợp của các biến, đoán biết trước được những rủi ro, những biến nào cần được chú trọng nhiều hơn. Dù có được sự tương quan giữa những nhân tố này, nhưng sẽ vẫn có một số nhân tố không đưa đến hình mẫu cuối cùng vì nó có ít giá trị so sánh với những biến số khác trong mô hình. Sử dụng các mô hình chấm điểm tín dụng, ngân hàng sẽ chấp nhận cho vay với những doanh nghiệp có điểm cao hơn một mức điểm sàn nào đó, từ chối những doanh nghiệp có điểm dưới mức điểm sàn và xem xét kỹ hơn hồ sơ của những người gần điểm sàn trước khi đưa ra quyết định cuối cùng. 

Kể cả một hệ thống chấm điểm tốt cũng không dự đoán chắc chắn khả năng hoàn trả vốn vay của doanh nghiệp nhưng nó cũng đưa ra được những dự đoán khá chính xác về sai sót mà một doanh nghiệp đi vay với những đặc điểm nhất định có thế mắc phải. Đế xây dựng một hình mẫu tốt, những người xây dựng phải có dữ liệu chính xác phản ánh khoản vay trong tất cả các giai đoạn vay, trong điều kiện kinh tế tốt và xấu.

\subsubsection{Mục đích của chấm điểm tín dụng}
Mục tiêu trước hết và quan trọng nhất của việc chấm điểm tín dụng là nhằm mục đích xác định được mức độ rủi ro mà ngân hàng phải đối mặt nếu như chấp nhận các khoản vay của khách hàng. Thông qua quá trình đánh giá xếp hạng của hệ thống xếp hạng tín dụng, NHTM có thể dự đoán được những sự khác biệt về mặt kinh tế giữa những gì mà người đi vay hứa sẽ thanh toán và những gì mà NHTM thực sự nhận được. 

Ngoài ra, việc đánh giá xếp hạng tín dụng còn giúp cho ngân hàng đạt được những mục tiêu cụ thể sau: 

\begin{itemize}
\item Hỗ trợ ngân hàng đưa ra quyết định về việc có chấp nhận hay từ chối các khoản vay, để từ đó có một chính sách tín dụng chính xác hơn. Chính sách này bao gồm việc xác định mức giá lãi vay, giới hạn lãi vay, các tài sản, điều kiện đảm bảo,... 
\item NHTM có thể đánh giá hiệu quả danh mục cho vay thông qua giám sát sự thay đổi dư nợ và phân loại nợ trong từng nhóm khách hàng đã được xếp hạng, qua đó điều chỉnh danh mục theo hướng ưu tiên nguồn lực vào những nhóm khách hàng an toàn. 
\item Phát hiện sớm các khoản tín dụng có khả năng bị tổn thất hay đi chệch hướng khỏi chính sách tín dụng của ngân hàng; xác định rõ khi nào cần có sự giám sát 
hoặc có các hoạt động điều chỉnh khoản tín dụng và ngược lại. 
\item Hỗ trợ cho ngân hàng trong quá trình thực hiện phân loại nợ và trích lập dự phòng rủi ro.
\end{itemize}
\subsubsection{Vai trò của chấm điểm tín dụng}

\section{Hiện trạng của các hệ thống chấm điểm tín dụng}
\subsection{Hiệp ước vốn Basel}
\subsubsection{Quá trình ra đời của Hiệp ước vốn Basel}

Uỷ ban Basel về giám sát ngân hàng (Basel Committee on Banking supervision - BCBS) được thành lập vào năm 1974 bởi một nhóm các Ngân hàng Trung ương và cơ quan giám sát của 10 nước phát triển (G10) tại thành phố Basel, Thụy Sỹ nhằm tìm cách ngăn chặn sự sụp đổ hàng loạt của các ngân hàng vào thập kỷ 80. Hiện nay, các thành viên của Ủy ban gồm đại diện ngân hàng trung ương hay cơ quan giám sát hoạt động ngân hàng của các nước: Anh, Bỉ, Canada, Đức, Hà Lan, Hoa Kỳ, Luxembourg, Nhật, Pháp, Tây Ban Nha, Thụy Điển, Thụy Sỹ và Ý. Ủy ban được nhóm họp 4 lần trong một năm.

Hội đồng thư ký của Ủy ban Basel được đề xuất bởi Ngân hàng Thanh toán Quốc tế ở Basel, gồm 15 thành viên là những nhà giám sát hoạt động ngân hàng chuyên nghiệp được biệt phái tạm thời từ các tổ chức tín dụng tài chính thành viên. Ủy ban Basel và các tiểu ban sẵn sàng đưa ra những lời tư vấn cho các cơ quan giám sát hoạt động ngân hàng ở tất cả các nước.

Ủy ban Basel không có bất kỳ một cơ quan giám sát nào và những kết luận của Uỷ ban này không có tính pháp lý và yêu cầu tuân thủ đối với việc giám sát hoạt động ngân hàng. Thay vào đó, Ủy ban Basel chỉ xây dựng và công bố những tiêu chuẩn và những hướng dẫn giám sát rộng rãi, đồng thời giới thiệu các báo cáo thực tiễn tốt nhất trong kỳ vọng rằng các tổ chức riêng lẻ sẽ áp dụng rộng rãi thông qua những sắp xếp chi tiết phù hợp nhất cho hệ thống quốc gia của chính họ. Theo cách này, Ủy ban khuyến khích việc áp dụng cách tiếp cận và các tiêu chuẩn chung mà không cố gắng can thiệp vào các kỹ thuật giám sát của các nước thành viên.

Ủy ban báo cáo thống đốc ngân hàng trung ương hay cơ quan giám sát hoạt động ngân hàng của nhóm G10. Từ đó tìm kiếm sự hậu thuẫn cho những sáng kiến của Ủy ban. Những tiêu chuẩn bao quát một dải rất rộng các vấn đề tài chính. Một mục tiêu quan trọng trong công việc của Ủy ban là thu hẹp khoảng cách giám sát quốc tế trên hai nguyên lý cơ bản là: (1) không ngân hàng nước ngoài nào được thành lập mà thoát khỏi sự giám sát; và (2) việc giám sát phải tương xứng. Để đạt được mục tiêu đề ra, từ năm 1975 đến nay, Ủy ban Basel đã ban hành rất nhiều văn bản, tài liệu liên quan đến vấn đề này.

Vào năm 1988, Ủy ban đã quyết định giới thiệu hệ thống đo lường vốn mà nó được đề cập như là Hiệp ước vốn Basel (the Basel Capital Accord) hay Basel I. Hệ thống này cung cấp khung đo lường rủi ro tín dụng với tiêu chuẩn vốn tối thiểu $8\%$. Basel I không chỉ được phổ biến trong các quốc gia thành viên mà còn được phổ biến ở hầu hết các nước khác có các ngân hàng hoạt động quốc tế. Đến năm 1996, Basel I được sửa đổi với rất nhiều điểm mới. Tuy vậy, Hiệp ước vẫn có khá nhiều điểm hạn chế.

Để khắc phục những hạn chế của Basel I, tháng 6/1999, Uỷ ban Basel đã đề xuất khung đo lường mới với 3 trụ cột chính: 
\begin{enumerate}
\item Yêu cầu vốn tối thiểu trên cơ sở kế thừa Basel I 
\item Sự xem xét giám sát của quá trình đánh giá nội bộ và sự đủ vốn của các tổ chức tài chính
\item Sử dụng hiệu quả của việc công bố thông tin nhằm làm lành mạnh kỷ luật thị trường như là một sự bổ sung cho các nỗ lực giám sát. Đến ngày 26/6/2004, bản Hiệp ước quốc tế về vốn Basel mới (Basel II) đã chính thức được ban hành.
\end{enumerate}


\subsubsection{Những điểm cơ bản của Basel I}

- Mục đích của Basel I: Củng cố sự ổn định của toàn bộ hệ thống ngân hàng quốc tế;
Thiết lập một hệ thống ngân hàng quốc tế thống nhất, bình đẳng nhằm giảm cạnh tranh không lành mạnh giữa các ngân hàng quốc tế.

- Tiêu chuẩn của Basel I: (1) Tỉ lệ vốn dựa trên rủi ro - “Tỉ lệ Cook”: tỉ lệ này được phát triển bởi BCBS với mục đích củng cố hệ thống ngân hàng quốc tế, đối tượng ban đầu là những ngân hàng hoạt động quốc tế, nhưng sau này đã được thực thi trên hơn 100 quốc gia. Theo tiêu chuẩn này, ngân hàng phải giữ lại lượng vốn bằng ít nhất $8\%$ của rổ tài sản, được tính toán theo nhiều phương pháp khác nhau và phụ thuộc vào độ rủi ro của chúng.

Tỉ lệ thoả đáng về vốn (CAR) = Vốn bắt buộc / Tài sản tính theo độ rủi ro gia quyền (RWA)

Theo đó, ngân hàng có mức vốn tốt là ngân hàng có CAR > $10\%$, có mức vốn thích hợp khi CAR > $8\%$, thiếu vốn khi CAR < $8\%$, thiếu vốn rõ rệt khi CAR < $6\%$ và thiếu vốn trầm trọng khi CAR < $2\%$.

(2) Vốn cấp 1, cấp 2 và cấp 3: Thành tựu cơ bản của Basel I là đã đưa ra được định nghĩa mang tính quốc tế chung nhất về vốn của ngân hàng và một cái gọi là tỷ lệ vốn an toàn của ngân hàng. Tiêu chuẩn này quy định:

\begin{center}
Vốn cấp 1 $\geq$ Vốn cấp 2 $+$ Vốn cấp 3
\end{center}

Vốn cấp 1 là lượng vốn dự trữ sẵn có và các nguồn dự phòng được công bố, như là khoản dự phòng cho các khoản vay, bao gồm: Vốn chủ sở hữu vĩnh viễn; Dự trữ công bố (Lợi nhuận giữ lại); Lợi ích thiểu số (minority interest) tại các công ty con, có hợp nhất báo cáo tài chính; Lợi thế kinh doanh (goodwill).

Vốn cấp 2 (Vốn bổ sung) gồm: Lợi nhuận giữ lại không công bố; Dự phòng đánh giá lại tài sản; Dự phòng chung/dự phòng thất thu nợ chung; Công cụ vốn hỗn hợp; Vay với thời hạn ưu đãi; Đầu tư vào các công ty con tài chính và các tổ chức tài chính khác.

Vốn Cấp 3 (Dành cho rủi ro thị trường) = Vay ngắn hạn

(3) Vốn tính theo rủi ro gia quyền:

RWA = Tổng (Tài sản x Mức rủi ro phân định cho từng tài sản trong bảng cân đối kế toán) + Tổng (Nợ tương đương x Mức rủi ro ngoại bảng)

Basel I đưa ra trọng số rủi ro gồm 4 mức: quốc gia $0\%$; ngân hàng $20\%$; doanh nghiệp $100\%$... Trọng số rủi ro không phản ánh độ nhạy cảm rủi ro trong mỗi loại này.

- Những thiếu sót của Basel I: Sau khi rủi ro tín dụng được thiết lập vào năm 1988, Uỷ ban Basel đã chuyển sự chú ý của họ sang rủi ro thị trường để phản ứng lại các hoạt động kinh doanh chuyên hữu ngày càng tăng của các ngân hàng thương mại và đến năm 1996, Basel I đã được sửa đổi với mục đích tính đến cả phí vốn đối với rủi ro thị trường.

Mặc dù vậy, Basel I vẫn có khá nhiều điểm hạn chế. Một trong những điểm hạn chế cơ bản của Basel I là không đề cập đến một loại rủi ro đang ngày càng trở nên phức tạp với mức độ ngày càng tăng lên, đó là rủi ro vận hành (không có yêu cầu vốn dự phòng rủi ro vận hành). Ngoài ra, còn một số điểm hạn chế khác, như: không phân biệt theo loại rủi ro, không có lợi ích từ việc đa dạng hóa...


\subsubsection{Những điểm cơ bản của Basel II}

- Mục tiêu của Basel II: Nâng cao chất lượng và sự ổn định của hệ thống ngân hàng quốc tế; Tạo lập và duy trì một sân chơi bình đẳng cho các ngân hàng hoạt động trên bình diện quốc tế; Đẩy mạnh việc chấp nhận các thông lệ nghiêm ngặt hơn trong lĩnh vực quản lý rủi ro.

Hai mục tiêu đầu của Basel II là những mục tiêu chủ chốt của Hiệp ước vốn Basel I. Mục tiêu cuối cùng là mới, đó là dấu hiệu của việc bắt đầu chuyển dần từ cơ chế điều tiết dựa trên tỷ lệ, mà đó chỉ là một phần của khung mới, hướng đến một sự điều tiết mà sẽ dựa nhiều hơn vào các số liệu nội bộ, thông lệ và các mô hình.

- Basel II sử dụng khái niệm“Ba trụ cột”:

(1) Trụ cột thứ I: liên quan tới việc duy trì vốn bắt buộc. Theo đó, tỷ lệ vốn bắt buộc tối thiểu (CAR) vẫn là $8\%$ của tổng tài sản có rủi ro như Basel I. Tuy nhiên, rủi ro được tính toán theo ba yếu tố chính mà ngân hàng phải đối mặt: rủi ro tín dụng, rủi ro vận hành (hay rủi ro hoạt động) và rủi ro thị trường. So với Basel I, cách tính chi phí vốn đối với rủi ro tín dụng có sự sửa đổi lớn, đối với rủi ro thị trường có sự thay đổi nhỏ, nhưng hoàn toàn là phiên bản mới đối với rủi ro vận hành. Trọng số rủi ro của Basel II bao gồm nhiều mức (từ $0\% -150\%$ hoặc hơn) và rất nhạy cảm với xếp hạng.

(2) Trụ cột thứ II: liên quan tới việc hoạch định chính sách ngân hàng, Basel II cung cấp cho các nhà hoạch định chính sách những “công cụ” tốt hơn so với Basel I. Trụ cột này cũng cung cấp một khung giải pháp cho các rủi ro mà ngân hàng đối mặt, như rủi ro hệ thống, rủi ro chiến lược, rủi ro danh tiếng, rủi ro thanh khoản và rủi ro pháp lý, mà hiệp ước tổng hợp lại dưới cái tên rủi ro còn lại (residual risk).

Basel II nhấn mạnh 4 nguyên tắc của công tác rà soát giám sát: Thứ nhất, các ngân hàng cần phải có một quy trình đánh giá được mức độ đầy đủ vốn nội bộ theo danh mục rủi ro và phải có được một chiến lược đúng đắn nhằm duy trì mức vốn đó. Thứ hai, các giám sát viên nên rà soát và đánh giá việc xác định mức độ vốn nội bộ và chiến lược của ngân hàng, cũng như khả năng giám sát và đảm bảo tuân thủ tỉ lệ vốn tối thiểu; giám sát viên nên thực hiện một số hành động giám sát phù hợp nếu họ không hài lòng với kết quả của quy trình này. Thứ ba, Giám sát viên khuyến nghị các ngân hàng duy trì mức vốn cao hơn mức tối thiểu theo quy định. Thứ tư, giám sát viên nên can thiệp ở giai đoạn đầu để đảm bảo mức vốn của ngân hàng không giảm dưới mức tối thiểu theo quy định và có thể yêu cầu sửa đổi ngay lập tức nếu mức vốn không được duy trì trên mức tối thiểu.

(3) Trụ cột thứ III: Các ngân hàng cần phải công khai thông tin một cách thích đáng theo nguyên tắc thị trường. Basel II đưa ra một danh sách các yêu cầu buộc các ngân hàng phải công khai thông tin, từ những thông tin về cơ cấu vốn, mức độ đầy đủ vốn đến những thông tin liên quan đến mức độ nhạy cảm của ngân hàng với rủi ro tín dụng, rủi ro thị trường, rủi ro vận hành và quy trình đánh giá của ngân hàng đối với từng loại rủi ro này.

Như vậy, quá trình phát triển của Basel và những Hiệp ước mà tổ chức này đưa ra, các ngân hàng thương mại càng ngày càng được yêu cầu hoạt động một cách minh bạch hơn, đảm bảo vốn phòng ngừa cho nhiều loại rủi ro hơn và do vậy, hy vọng sẽ giảm thiểu được rủi ro.

\subsubsection{Ưu điểm của Basel II so với Basel I}

- Về cấu trúc và nội dung: Basel I tập trung vào một giải pháp quản lý rủi ro duy nhất là “yêu cầu vốn tối thiểu”. Trong khi, Basel II tập trung nhiều hơn vào các phương pháp nội bộ của chính ngân hàng, đánh giá hoạt động thanh tra, giám sát và kỷ luật trên nguyên tắc thị trường. Do đó, quyền lực của các nhà quản lý quốc gia được tăng lên bởi họ cần phải đánh giá sự đủ vốn của ngân hàng có tính đến đặc điểm rủi ro cụ thể của nó.

- Về tính linh động của ứng dụng: Basel I quy định chung một chọn lựa cho tất cả các ngân hàng. Basel II linh hoạt hơn với một danh sách các phương pháp, các biện pháp khuyến khích để các nhà quản lý quốc gia và các ngân hàng chọn lựa.

- Về tính nhạy cảm với rủi ro: Basel I đo đạc rủi ro quá sơ bộ. Basel II nhạy cảm hơn với rủi ro thông qua độ nhạy cảm của yêu cầu vốn đối với mức độ rủi ro tăng lên và sự công khai bắt buộc một cách chi tiết về độ nhạy cảm rủi ro và chính sách rủi ro.

- Về trọng số rủi ro: Basel I quy định từ 0 – 100 và ưu đãi hơn với các nước thuộc Tổ chức hợp tác và phát triển kinh tế (OECD- Organisation for Economic Co-operation and Development). Basel II quy định từ 0 - 150 hoặc hơn và không có đặc quyền nào, bao gồm cả phân cấp bên trong và bên ngoài.

- Về kỹ thuật giảm rủi ro tín dụng: Basel I chỉ hỗ trợ và đảm bảo. Basel II thừa nhận về kỹ thuật giảm thiểu rủi ro tốt hơn, đưa ra nhiều kỹ thuật hơn như hỗ trợ, đảm bảo, phái sinh tín dụng, lập mạng lưới vị thế (position netting).

\subsection{Hệ thống chấm điểm tín dụng ở Việt Nam}
Căn cứ vào Điều 7 của Quyết định 493/2005/QĐ-NHNN và các quy định có liên quan của từng ngân hàng nhằm xác lập quy trình xếp hạng tín dụng, một quy trình 
xếp hạng tín dụng bao gồm các bước cơ bản như sau : 
\begin{itemize}
\item Thu thập thông tin liên quan đến các chỉ tiêu sử dụng trong phân tích đánh giá, thông tin xếp hạng của các tổ chức tín nhiệm khác liên quan đến đối tượng xếp hạng. Trong quá trình thu thập thông tin, ngoài những thông tin do chính khách hàng cung cấp, cán bộ thẩm định phải sử dụng nhiều nguồn thông tin khác từ các phương tiện thông tin đại chúng, thông tin từ trung tâm tín dụng của ngân hàng, thông tin từ CIC, ...
\item Phân tích bằng mô hình để kết luận về mức xếp hạng. Sử dụng các chỉ tiêu nhân thân và quan hệ với ngân hàng. Mức xếp hạng cuối cùng được quyết định sau khi tham khảo ý kiến Hội đồng xếp hạng. Trong quá trình xếp hạng tín dụng của các NHTM thì kết quả xếp hạng không được công bố rộng rãi.
\item Theo dõi tình trạng tín dụng của đối tượng được xếp hạng để điều chỉnh mức xếp hạng. các thông tin điều chỉnh được lưu giữ. Tổng hợp kết quả xếp hạng so sánh với thực tế rủi ro xảy ra, và dựa trên tần suất phải điều chỉnh mức xếp hạng đã thực hiện đối với khách hàng để xem xét điều chỉnh mô hình xếp hạng.
\end{itemize}

Việc tiếp cận Basel II đòi hỏi kỹ thuật phức tạp và chi phí khá cao. Đối với một nước có hệ thống ngân hàng mới đang ở giai đoạn phát triển ban đầu như Việt Nam, việc áp dụng Basel II gặp nhiều khó khăn, thách thức và mất nhiều thời gian. Tuy nhiên, trước xu thế hội nhập và mở cửa thị trường dịch vụ tài chính - ngân hàng với nhiều loại hình dịch vụ ngân hàng mới, việc áp dụng Basel II tại Việt Nam là yêu cầu cấp thiết nhằm tăng cường năng lực hoạt động và giảm thiểu rủi ro đối với các ngân hàng thương mại (NHTM).

Sau khi Việt Nam gia nhập WTO, NHNN Việt Nam và các TCTD Việt Nam đã có nhiều nỗ lực trong việc hoàn thiện hệ thống pháp lý về tiền tệ và hoạt động ngân hàng cũng như nâng cao năng lực quản trị điều hành, đặc biệt là năng lực quản trị rủi ro của các NHTM tiến dần từng bước đến các thông lệ và chuẩn mực quốc tế. Theo đó, việc từng bước áp dụng các chuẩn mực của Basel II được đặc biệt chú trọng, nhất là sau cuộc khủng hoảng tài chính và suy thoái kinh tế toàn cầu thời gian qua.

Về phía các tổ chức tín dụng Việt Nam, Basel II đã có ảnh hưởng lớn trong việc nâng cao năng lực quản trị điều hành, nhất là năng lực quản lý rủi ro. Bên cạnh việc tuân thủ các quy định bắt buộc của NHNN, các TCTD cũng đang rất nỗ lực để hoàn thiện hơn nữa hệ thống quản trị rủi ro của ngân hàng mình cho phù hợp với điều kiện hoạt động cụ thể của mỗi ngân hàng và từng bước tiếp cận với các chuẩn mực của Basel II.

Mặc dù được coi như một cơ chế quan trọng để đẩy mạnh cải cách và củng cố toàn bộ công tác điều hành trong lĩnh vực tài chính, nhưng cuộc khủng hoảng tài chính hiện tại đã cho thấy những thiếu sót, bất cập của Basel II. Một số thiếu sót cơ bản của Basel II là thiếu yêu cầu về phí vốn thanh khoản, quá tin cậy vào cơ quan xếp hạng tín dụng và bản chất có tính chu kỳ của nó.
