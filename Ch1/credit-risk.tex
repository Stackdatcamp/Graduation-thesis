\section{Tổng quan về rủi ro tín dụng và vấn đề quản trị rủi ro trong ngân hàng}
\subsection{Khái niệm về rủi ro và chấm điểm tín dụng}
\subsubsection{Tổng quan về rủi ro tín dụng}
Là rủi ro do một khách hàng hay một nhóm khách hàng vay vốn không trả được nợ cho Ngân hàng. Trong kinh doanh Ngân hàng rủi ro tín dụng là loại rủi ro lớn nhất, thường xuyên xảy ra và gây hậu quả nặng nề có khi dẫn đến phá sản Ngân hàng.

Ngày nay, nhu cầu về vốn để mở rộng sản xuất kinh doanh, cải tiến trang thiết bị kỹ thuật, nâng cao công nghệ và các nhu cầu phục vụ sản xuất kinh doanh luôn tăng lên. Để đáp ứng nhu cầu này, các NHTM cũng phải luôn mở rộng quy mô hoạt động tín dụng, điêu đó có nghĩa là rủi ro tín dụng cũng phát sinh nhiều hơn.

Rủi ro tín dụng là loại rủi ro phức tạp nhất, việc quản lý và phòng ngừa nó rất khó khăn, nó có thể xảy ra ở bất cứ đâu, bất cứ lúc nào... Rủi ro tín dụng nếu không được phát hiện và sử lý kịp thời sẽ nảy sinh các rủi ro khác.

\subsubsection{Nguyên nhân dẫn đến rủi ro tín dụng}
\paragraph{Nguyên nhân từ phía Ngân hàng.}
Thực tế kinh doanh của Ngân hàng trong thời gian qua cho thấy rủi ro tín dụng xảy ra là do những nguyên nhân sau:

- Ngay hàng đưa ra chính sách tín dụng không phù hợp với nền kinh tế và thể lệ cho vay còn sơ hở để khách hàng lợi dụng chiếm đoạt vốn của Ngân hàng.

- Do cán bộ Ngân hàng chưa chấp hành đúng quy trình cho vay như: không đánh giá đầy đủ chính xác khách hàng trước khi cho vay, cho vay khống, thiếu tài sản đảm bảo, cho vay vượt tỷ lệ an toàn. Đồng thời cán bộ Ngân hàng không kiểm tra, giám sát chặt chẽ về tình hình sử dụng vốn vay của khách hàng.

- Do trình độ nghiệp vụ của cán bộ tín dụng còn nên việc đánh giá các dự án, hồ sơ xin vay còn chưa tốt, còn xảy ra tình trạng dự án thiếu tính khả thi mà vẫn cho vay.

- Cán bộ Ngân hàng còn thiếu tinh thần trách nhiệm, vi phạm đạo đức kinh doanh như: thông đồng với khách hàng lập hồ sơ giả để vay vốn, xâm tiêu khi giải ngân hay thu nợ, đôi khi còn nể nang trong quan hệ khách hàng.

- Ngân hàng đôi khi quá chú trọng về lợi nhuận, đặt những khoản vay có lợi nhuân cao hơn những khoản vay lành mạnh.

- Do áp lực cạnh tranh với các Ngân hàng khác.

- Do tình trạng tham nhũng, tiêu cực diễn ra trong nội bộ Ngân hàng

\paragraph{Nguyên nhân từ phía khách hàng.}

- Người vay vốn sử dụng vốn vay sai mục đích, sử dụng vào các hoạt động có rủi ro cao dẫn đến thua lỗ không trả được nợ cho Ngân hàng.

- Do trình độ kinh doanh yếu kếm, khả năng tổ chức điều hành sản xuất kinh doanh của lãnh đạo còn hạn chế.

- Doanh nghiệp vay ngắn hạn để đầu tư vào tài sản lưu động và cố định.

- Doanh nghiệp sản xuất kinh doanh thiếu sự linh hoạt, không cải tiến quy trình công nghệ, không trang bị máy móc hiện đại, không thay đổi mẫu mã hoặc nghiên cứu nâng cao chất lượng sản phẩm...dẫn tới sản phẩm sản xuất ra thiếu sự cạnh tranh, bị ứ đọng trên thị trường khiến cho doanh nghiệp không có khả năng thu hồi vốn trả nợ cho Ngân hàng.

- Do bản thân doanh nghiệp có chủ ý lừa gạt, chiếm dụng vốn của Ngân hàng, dùng một loại tài sản thế chấp đi vay nhiều nơi, không đủ năng lực pháp nhân.

\paragraph{Nguyên nhân khác.}

- Do sự thay đổi bất thường của các chính sách, do thiên tai bão lũ, do nền kinh tế không ổn định.... khiến cho cả Ngân hàng và khách hàng không thể ứng phó kịp.

- Do môi trường pháp lý lỏng lẻo, thiếu đồng bộ, còn nhiều sơ hở dẫn tới không kiểm soát được các hiện tượng lừa đảo trong việc sử dụng vốn của khách hàng.

- Do sự biến động về chính trị - xã hội trong và ngoài nước gây khó khăn cho doanh nghiệp dẫn tới rủi ro cho Ngân hàng.

- Ngân hàng không theo kịp đà phát triển của xã hội, nhất là sự bất cập trong trình độ chuyên môn cũng như công nghệ Ngân hàng.

- Do sự biến động của kinh tế như suy thoái kinh tế, biến động tỷ giá, lạm phát gia tăng ảnh hưởng tới doanh nghiệp cũng như Ngân hàng.

- Sự bất bình đẳng trong đối sử của Nhà nước dành cho các NHTM khác nhau.

- Chính sách Nhà nước chậm thay đổi hoặc chưa phù hợp với tình hình phát triển đất nước.

\subsubsection{Sự cần thiết phải phòng ngừa rủi ro tín dụng.}

\paragraph{Đối với bản thân Ngân hàng}
Các nhà kinh tế thường gọi Ngân hàng là “ngành kinh doanh rủi ro”. Thực tế đã chứng minh không một ngành nào mà khả năng dẫn đến rủi ro lại lớn như trong lĩnh vực kinh doanh tiền tệ- tín dụng. Ngân hàng phải gánh chịu những rủi ro không những do nguyên nhân chủ quan của mình, mà còn phải gánh chịu những rủi ro khách hàng gây ra. Vì vậy “rủi ro tín dụng của Ngân hàng không những là cấp số cộng mà có thể là cấp số nhân rủi ro của nền kinh tế”.

Khi rủi ro xảy ra, trước tiên lợi nhuận kinh doanh của Ngân hàng sẽ bị ảnh hưởng. Nếu rủi ro xảy ra ở mức độ nhỏ thì Ngân hàng có thể bù đắp bằng khoản dự phòng rủi ro ( ghi vào chi phí ) và bằng vốn tự có, tuy nhiên nó sẽ ảnh hưởng trực tiếp tới khả năng mở rộng kinh doanh của Ngân hàng. Nghiêm trọng hơn, nếu rủi ro xảy ra ở mức độ lớn, nguồn vốn của Ngân hàng không đủ bù đắp, vốn khả dụng bị thiếu, lòng tin của khách hàng giảm tất nhiên sẽ dẫn tới phá sản Ngân hàng. Vì vậy việc phòng ngừa và hạn chế rủi ro tín dụng là một việc làm cần thiết đối với các NHTM.

\paragraph{Đối với nền kinh tế}
Trong nền kinh tế thị trường, hoạt động kinh doanh của Ngân hàng liên quan đến rất nhiều các thành phần kinh tế từ cá nhân, hộ gia đình, các tổ chức kinh tế cho tới các tổ chức tín dụng khác. Vì vậy, kết quả kinh doanh của Ngân hàng phản ánh kết quả sản xuất kinh doanh của nền kinh tế và đương nhiên nó phụ thuộc rất lớn vào tình hình tổ chức sản xuất kinh doanh của các doanh nghiệp và khách hàng. Hoạt động kinh doanh của Ngân hàng không thể có kết quả tốt khi hoạt động kinh doanh của nền kinh tế chưa tốt hay nói cách khác hoạt động kinh doanh của Ngân hàng sẽ có nhiều rủi ro khi hoạt động kinh tế có nhiều rủi ro. Rủi ro xảy ra dẫn tới tình trạng mất ổn định trên thị trường tiền tệ, gây khó khăn cho các doanh nghiệp sản xuất kinh doanh, làm ảnh hưởng tiêu cực đối với mnền kinh tế và đời sống xã hội. Do đó, phòng ngừa và hạn chế rủi ro tín dụng không những là vấn đề sống còn với Ngân hàng mà còn là yêu cầu cấp thiết của nền kinh tế góp phần vào sự ổn định và phát triển của toàn xã hội.



\subsection{Tổng quan về chấm điểm tín dụng tại các ngân hàng}
\subsubsection{Khái niệm về chấm điểm tín dụng}
CĐTD là một phương thức đế đánh giá rủi ro của những đổi tượng đi vay. 
Theo đó ,ngân hàng sử dụng phương pháp thông kê, nghiên cứu dữ liệu đế đánh giá rủi ro của người vay. 
Phương pháp này đưa ra “điểm” mà ngân hàng có thế sử dụng đế xếp loại những người xin vay xét về độ mạo hiểm. Đẻ tạo dựng một hình mẫu chấm điểm, hay một “bảng điểm”, thì những nhà kinh tế phân tích những dữ liệu trong quá khứ về sự thực hiện các khoản vay trước đó đế quyết định những đặc điếm của những người đi vay nào là hữu ích trong việc phỏng đoán xem liệu khoản vay đó có phát huy tốt tác dụng không. Một hình mẫu được thiết kế tốt sẽ đưa ra tỷ lệ điểm cao nhiều hơn cho những người đi vay có khả năng sử dụng vốn vay hiệu quả và ngược lại, tỷ lệ phần trăm điểm thấp nhiều hơn cho những người đi vay mà những khoản vay ít phát huy tác dụng. Nhưng không có hình mẫu nào là hoàn hảo, cho nên đôi khi có những đối tác không tốt lại nhận được điểm cao hơn. Thông tin của những người đi vay được thu nhận từ những bản đăng ký và từ bưu cục tín dụng những dữ liệu như thu nhập hàng tháng của doanh nghiệp đi vay, khoản nợ đọng, tài sản tài chính, khoản thời gian mà doanh nghiệp hoạt động trong lĩnh vực kinh doanh của mình, liệu doanh nghiệp đã tùng phạm lỗi trong một khoản vay trước đó hay không, liệu và loại tài khoản gân hàng mà doanh nghiệp đi vay có là tất cả những yếu tố tiềm năng có khả năng đánh giá được khoản vay mà có thể được sử dụng trong bảng điểm. Phân tích tổng hợp liên quan đến khoản vay từ những biến số ở trên được sử dụng để tìm ra sự kết hợp của những nhân tố, đoán biết trước được những rủi ro, những nhân tố nào cần được chú trọng nhiều hơn. Dù có được sự tương quan giữa những nhân tố này, nhưng sẽ vẫn có một số nhân tố không đưa đến hình mẫu cuối cùng vì nó có ít giá trị so sánh với những biến sô khác trong hình mẫu. Trên thực tế theo công ty Issac and Company,Inc., người dẫn đầu trong việc phát triến hình mẫu chấm điếm này, 50 - 60 biến sổ có thế được xem xét khi phát triển hình mẫu thông thường, nhưng chỉ 8 - 12 có thể đưa đến bảng điểm có thế phỏng đoán tốt nhất. Anthony Sauder, một nhà kinh tế học của Mỹ sử dụng 48 nhân tố để đánh giá xác suất lỗi tín dụng trong phần lớn (nhưng không phải tất cả) các hệ thống chấm điểm, điểm cao hơn ám chỉ ít rủi ro hơn, ngân hàng cho vay sẽ đặt điểm sàn dựa trên tỉ lệ mạo hiểm mà ngân hàng đó sẵn sàng chấp nhận. Hoàn toàn tuân thủ theo hình mẫu đó, ngân hàng cho vay sẽ chấp nhận cho vay với những doanh nghiệp có điểm trên điểm sàn, và từ chối những doanh nghiệp dưới điếm sàn. Mặc dù có nhiều ngân hàng có thế xem xét kỹ hơn hồ sơ của những người gần điểm sàn trước khi đưa ra quyết định cuối cùng. 

Kể cả một hệ thống chấm điểm tốt cũng không dự đoán chắc chắn khả năng hoàn trả vốn vay của doanh nghiệp nhưng nó cũng đưa ra được những dự đoán khá chính xác về sai sót mà một doanh nghiệp đi vay với những đặc điểm nhất định có thế mắc phải. Đế xây dựng một hình mẫu tốt, những người xây dựng phải có dữ liệu chính xác phản ánh khoản vay trong cả giai đoạn, trong điều kiện kinh tế tốt và xấu.

\subsubsection{Mục đích của chấm điểm tín dụng}
\subsubsection{Vai trò của chấm điểm tín dụng}

\subsection{Hiện trạng của các hệ thống chấm điểm tín dụng}
\subsubsection{Các hệ thống chấm điểm tín dụng trên thế giới}
\subsubsection{Các hệ thống chấm điểm tín dụng ở Việt Nam}

