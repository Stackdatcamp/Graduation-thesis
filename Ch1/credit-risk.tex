\section{Tổng quan về rủi ro tín dụng và vấn đề quản trị rủi ro trong ngân hàng}
\subsection{Rủi ro và các khái niệm liên quan trong ngân hàng}
\subsubsection{Tín dụng}
\begin{comment}
 %%% http://dspace.elib.ntt.edu.vn/dspace/bitstream/123456789/4766/1/Le%20Nguyen%20Phuong%20Ngoc.pdf


Theo quyết định 1627/2001/QĐ-NHNN ngày 31/12/2001 của Thống Đốc Ngân
Hàng Nhà Nước thì cho vay là một hình thức cấp tín dụng, theo đó tổ chức tín dụng
giao cho khách hàng sử dụng một khỏan tiền để sử dụng vào một mục đích và thời gian
nhất định theo thỏa thuận với nguyên tắc có hòan trả cả gốc và lãi.
Như vậy tín dụng ngân hàng là quan hệ chuyển nhượng quyền sử dụng vốn từ
ngân hàng tới khách hàng theo những điều kiện ràng buộc nhất định. Cũng như quan hệ
tín dụng khác, tín dụng ngân hàng chứa đựng ba nội dung:
Có sự chuyển nhượng quyền sử dụng vốn từ người sở hữu sang cho người sử
dụng.
Sự chuyển nhượng này có thời hạn cụ thể.
Sự chuyển nhượng này có kèm theo chi phí. 

%%% Bài của Cheesé
Tín dụng DNNVV được hiểu đơn giản là hoạt động tín dụng hay các khoản tín dụng mà các NHTM cấp cho khách hàng là các DNNVV. Như vậy, tín dụng DNNVV là một bộ phận trong hoạt động tín dụng của NHTM, được phân chia dựa trên tiêu chí là đối tượng khách hàng.
Khái niệm chung về tín dụng ngân hàng là giao dịch về tài sản giữa ngân hàng (TCTD) với bên đi vay (là các tổ chức kinh tế, cá nhân trong nền kinh tế) trong đó ngân hàng (TCTD) chuyển giao tài sản cho bên đi vay sử dụng trong một thời gian nhất định theo thoả thuận, và bên đi vay có trách nhiệm hoàn trả vô điều kiện cả vốn gốc và lãi cho ngân hàng (TCTD) khi đến hạn thanh toán.
Theo Luật các tổ chức tín dụng 2010 của Việt Nam: “Cấp tín dụng là việc thỏa thuận để tổ chức, cá nhân sử dụng một khoản tiền hoặc cam kết cho phép sử dụng một khoản tiền theo nguyên tắc có hoàn trả bằng nghiệp vụ cho vay, chiết khấu, cho thuê tài chính, bao thanh toán, bảo lãnh ngân hàng và các nghiệp vụ cấp tín dụng khác”.
Trong hoạt động của các NHTM Việt Nam hiện nay, hoạt động tín dụng là nghiệp vụ nền tảng, truyền thống, đóng vai trò là một trong những hoạt động tạo ra lợi nhuận lớn nhất cũng như chiếm tỷ trọng cao trong cơ cấu tài sản của ngân hàng. Tín dụng là hoạt động tài trợ vốn của ngân hàng cho khách hàng. Hoạt động này luôn đóng vai trò quan trọng nhất, quyết định sự thành bại của ngân hàng. Nó cũng là hoạt động cơ bản, sinh lời chủ yếu của ngân hàng.


%%% https://thebank.vn/posts/8659-the-nao-la-tin-dung-ngan-hang#sthash.FC44lnZU.dpuf
Tín dụng là khái niệm thể hiện mối quan hệ giữa người cho vay và người vay. Trong quan hệ này, người cho vay có nhiệm vụ chuyển giao quyền sử dụng tiền hoặc hàng hoá cho vay cho người đi vay trong một thời gian nhất định. Người đi vay có nghĩa vụ trả số tiền hoặc giá trị hàng hoá đã vay khi đến hạn trả nợ có kèm hoặc không kèm theo một khoản lãi.

Tín dụng ngân hàng là quan hệ tín dụng giữa ngân hàng, các tổ chức tín dụng khác, với các nhà doanh nghiệp và cá nhân.
\end{comment}

\subsubsection{Rủi ro tín dụng trong ngân hàng}
\begin{comment}
%%% Cheessệ

Rủi ro tín dụng trong hoạt động NH là khả năng xảy ra tổn thất trong hoạt động NH do khách hàng không thực hiện hoặc không có khả năng thực hiện nghĩa vụ của mình theo cam kết. 
Rủi ro tín dụng được hiểu một cách đơn giản nhất đó là rủi ro không thu hồi được nợ khi đến hạn. Xuất phát từ hoạt động tín dụng khi khách hàng vay vi phạm các điều kiện của hợp đồng tín dụng làm giảm hay mất giá trị của tài sản có. Rủi ro tín dụng phát sinh trong trường hợp NH không thu được đầy đủ cả gốc lẫn lãi của khoản vay, hoặc là việc thanh toán nợ gốc và lãi không đúng kỳ hạn.

\end{comment}

\subsection{Quản trị rủi ro trong ngân hàng}
\subsubsection{Hoạt động quản trị rủi ro trong ngân hàng}
\subsubsection{Quản trị rủi ro tín dụng trong lịch sử}
\subsubsection{}
\section{Thực trạng của việc chấm điểm tín dụng tại Việt Nam}

\section{Kết luận}
