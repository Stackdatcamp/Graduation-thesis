\message{ !name(def-logistics.rnw.tex)}
\message{ !name(def-logistics.rnw) !offset(-2) }
\subsubsection{Khái niệm}

Mô hình hồi quy Logistic (hay logit) được dùng để nghiên cứu mối quan hệ giữa xác suất của các biến nhị phân hoặc phân loại và các biến giải thích khác. Hướng tiếp cận của mô hình Logistic cho bài toán phân loại là bằng cách ước lượng giá trị xác suất $P(y = 1|X)$ như sau:

{\large
$$
P(y = 1|X) = \frac{e^{\beta_0 + \sum\beta_1X_1 +\beta_2X_2 + \ldots + \beta_nX_n}}{1 + e^{\beta_0 + \sum\beta_1X_1 +\beta_2X_2 + \ldots + \beta_nX_n}}
$$
}

Với $y$ là biến dùng để phân loại, chỉ nhận hai giá trị 0 hoặc 1, $X$ là các vector của biến độc lập, $\beta_0$, $\beta_1$, $\beta_2$, ..., $\beta_n$ là các hệ số cần ước lượng. 

Hay còn được viết dưới dạng:

{\large
$$
\log\frac{P(y = 1|X)}{P(y = 0|X)} = \beta_0 + \beta_1X_1 +\beta_2X_2 + \ldots + \beta_nX_n
$$
}

% Trong đó giá trị $\log [P(y = 1|X)/P(y = 0|X) ]$ còn được gọi là 
 
%If some of the independent variables are discrete, nominal scale variables such as race, sex, treatment group, and so forth, it is inappropriate to include them in the model as if they were interval scale variables. The numbers used to represent the various levels of these nominal scale variables are merely identifiers, and have no numeric significance. In this situation, the method of choice is to use a collection of design variables (or dummy variables). Suppose, for example, that one of the independent variables is race, which has been coded as “white,” “black,” and “other.” In this case, two design variables are necessary. One possible coding strategy is that when the respondent is “white,” the two design variables, D 1 and D 2 , would both be set equal to zero; when the respondent is “black,” D 1 would be set equal to 1 while D 2 would still equal 0; when the race of the respondent is “other,” we would use D 1 = 0 and D 2 = 1.
Trong trường hợp các biến độc lập là biến phân loại không so sánh được (ví dụ: Giới tính, dân tộc, v..v..) chúng ta đưa các biến này vào mô hình bằng cách sử dụng một nhóm các biến giả tương ứng với từng giá trị khác nhau của biến phân loại.

\subsubsection{Ước lượng mô hình logit}
a, Ước lượng hợp lý tối đa

\vspace{5mm}

%Logistic regression models are usually fit by maximum likelihood, using the conditional likelihood of G given X. Since Pr(G|X) completely specifies the conditional distribution, the multinomial distribution is appropriate. The log-likelihood for N observations is

Các hệ số $\beta$ thường được ước lượng bằng phương pháp ước lượng hợp lý tối đa \parencite{hosmer2013applied}, sử dụng hàm hợp lý có điều kiện $G$ đối với mỗi giá trị của $X$. Hàm hợp lý logarit cho $N$ quan sát được viết như sau:

{\large

$$
L(\theta) = \sum_{i = 1}^N P_{g_i} (x_i; \theta)
$$
}
, với $p_k(x_i;\theta) = P(G = k|X = x_i; \theta)$.

Trong trường hợp biến phụ thuộc $Y$ chỉ có 2 giá trị: $(0, 1)$, ta có thể mã hóa 2 nhóm của $g_i$ thành$y_i = 1$ khi $g_i = 1$ và $y_i = 0$ khi $g_i = 2$. Khi đó logarit của hàm hợp lý có thể viết lại như  sau: 

{\large
$$
L(\beta) = \sum_{i = 1}^N \{  y_i log (x_i ; \beta) + (1- y_i)log (x_i; \beta) \}
$$
}

Tối ưu hóa hàm $L$ sẽ cho chúng ta ước lượng hợp lý tối đa cho các hệ số $\beta$ trong mô hình.

b, Ràng buộc L1 hay mô hình Lasso

\vspace{5mm}

Một vấn đề mô hình logit hay gặp phải đó là hiện tượng đa cộng tuyến giữa các biến khi số lượng biến $p$ tăng lên. Hậu quả của hiện tượng này là các ước lượng cho hệ số $\beta$ thường là có sai số lớn, mặc dù ước lượng vẫn là không chệch. Nói cách khác, các giá trị $\beta$ ước lượng được thường có hiệu quả kém khi áp dụng trên mẫu mới, mặc dù mô hình vẫn có độ chính xác cao khi áp dụng trên bộ số liệu mẫu dùng để ước lượng ra mô hình. 

Để xử lý vấn đề này, chúng ta có thể áp dụng nhiều phương pháp để loại biến ra khỏi mô hình, hoặc sử dụng các phương pháp ước lượng khác mà các biến có ý nghĩa thống kê thấp bị loại ra khỏi mô hình trong quá trình ước lượng.

Phương pháp Lasso là một cải tiến của các mô hình tuyến tính, trong mô hình này, chúng ta áp dụng thêm ràng bộc L1 đối với hàm hợp lý tối đa. Áp dụng với mô hình Logit, thay vì tối ưu hàm hợp lý tối đa, chúng ta tối ưu:

{\large
$$
\max_{\beta_0, \beta} \left \{ \sum_{i = 1}^N [ y_i log (x_i ; \beta) + (1- y_i)log (x_i; \beta)] - \lambda \sum_{j = 1}^p|\beta_j|\right \}
$$
}

%As with the lasso, we typically do not penalize the intercept term, and stan-dardize the predictors for the penalty to be meaningful. Criterion (4.31) is126 concave, and a solution can be found using nonlinear programming meth-ods (Koh et al., 2007, for example). Alternatively, using the same quadratic approximations that were used in the Newton algorithm in Section 4.4.1, we can solve (4.31) by repeated application of a weighted lasso algorithm. Interestingly, the score equations [see (4.24)] for the variables with non-zero coefficients have the form
Sử dụng các giá trị khác nhau của tham số $\lambda$, phương pháp lasso thu nhỏ giá trị ước lượng của các $\beta$ so với phương pháp tối đa hóa hàm hợp lý truyền thống. Vì các giá trị thu nhỏ của $\beta$ có thể giảm về 0 và loại ra khỏi mô hình nếu $\lambda$ đủ lớn, phương pháp lasso có thể được dùng để thay thế cho việc chọn biến trong các mô hình đa biến.

Đồng thời đối với lasso, ta thường không ràng buộc các hệ số chặn, và các biến giải thích phải được chuẩn hóa để ràng buộc chung cho các $\beta$ là có ý nghĩa. Vì hàm tối ưu của chúng ta là hàm lõm, lời giải có thể tính sử dụng các phương pháp phi tuyến. %need citation


\subsubsection{Diễn giải kết quả ước lượng mô hình}

\message{ !name(def-logistics.rnw.tex) !offset(-79) }
