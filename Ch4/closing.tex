
%%% Local Variables:
%%% mode: latex
%%% TeX-master: "../Main.Rnw"
%%% End:

Trong đề tài này, chúng ta đã trình bày sơ lược về các khái niệm trong lĩnh vực chấm điểm tín dụng trong ngân hàng và đi sâu vào một tình huống nghiên cứu: xây dựng một mô hình phân loại phù hợp cho bộ số liệu về các khách hàng sử dụng thẻ tín dụng.

Trong tình huống nghiên cứu này, chúng ta sử dụng hai lớp mô hình khác biệt, đều là các kỹ thuật đang được sử dụng phổ biến trên thế giới không chỉ trong lĩnh vực quản lý rủi ro tín dụng trong ngân hàng mà còn trong nhiều bài toán ứng dụng khác. 

Tuy kết quả của các mô hình có sự khác biệt, chúng ta rút ra kết luận rằng trong việc xây dựng mô hình chấm điểm tín dụng, chúng ta nên ứng dụng một cách linh hoạt nhiều mô hình khác nhau để phục vụ cho mục đích của việc nghiên cứu. Các mô hình có mức độ hiệu cao thường khó có thể hiểu được ý nghĩa chúng, trong khi các mô hình giúp chúng ta hiểu được các mối quan hệ trong thực tế thì chưa chắc đã có hiệu quả cao.

Bên cạnh những kết quả đã đạt được trong bài, nên lưu ý rằng vẫn còn nhiều cách để phát triển các mô hình đã có để có thể ứng dụng trong nhiều trường hợp cụ thể khác nhau, nhằm tối ưu hóa các lợi ích có được từ việc xây dựng mô hình. Đồng thời, bộ số liệu mẫu dùng để nghiên cứu vẫn còn khá đơn giản do số biến không nhiều. Để có thể xây dựng được các mô hình hiệu quả hơn, cần có sự đầu tư nghiên cứu cũng như kiến thức về chuyên môn và hoàn cảnh nội bộ của từng ngân hàng.