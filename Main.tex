\documentclass[a4paper]{report}\usepackage[]{graphicx}\usepackage[]{color}
%% maxwidth is the original width if it is less than linewidth
%% otherwise use linewidth (to make sure the graphics do not exceed the margin)
\makeatletter
\def\maxwidth{ %
  \ifdim\Gin@nat@width>\linewidth
    \linewidth
  \else
    \Gin@nat@width
  \fi
}
\makeatother

\definecolor{fgcolor}{rgb}{0.345, 0.345, 0.345}
\newcommand{\hlnum}[1]{\textcolor[rgb]{0.686,0.059,0.569}{#1}}%
\newcommand{\hlstr}[1]{\textcolor[rgb]{0.192,0.494,0.8}{#1}}%
\newcommand{\hlcom}[1]{\textcolor[rgb]{0.678,0.584,0.686}{\textit{#1}}}%
\newcommand{\hlopt}[1]{\textcolor[rgb]{0,0,0}{#1}}%
\newcommand{\hlstd}[1]{\textcolor[rgb]{0.345,0.345,0.345}{#1}}%
\newcommand{\hlkwa}[1]{\textcolor[rgb]{0.161,0.373,0.58}{\textbf{#1}}}%
\newcommand{\hlkwb}[1]{\textcolor[rgb]{0.69,0.353,0.396}{#1}}%
\newcommand{\hlkwc}[1]{\textcolor[rgb]{0.333,0.667,0.333}{#1}}%
\newcommand{\hlkwd}[1]{\textcolor[rgb]{0.737,0.353,0.396}{\textbf{#1}}}%
\let\hlipl\hlkwb

\usepackage{framed}
\makeatletter
\newenvironment{kframe}{%
 \def\at@end@of@kframe{}%
 \ifinner\ifhmode%
  \def\at@end@of@kframe{\end{minipage}}%
  \begin{minipage}{\columnwidth}%
 \fi\fi%
 \def\FrameCommand##1{\hskip\@totalleftmargin \hskip-\fboxsep
 \colorbox{shadecolor}{##1}\hskip-\fboxsep
     % There is no \\@totalrightmargin, so:
     \hskip-\linewidth \hskip-\@totalleftmargin \hskip\columnwidth}%
 \MakeFramed {\advance\hsize-\width
   \@totalleftmargin\z@ \linewidth\hsize
   \@setminipage}}%
 {\par\unskip\endMakeFramed%
 \at@end@of@kframe}
\makeatother

\definecolor{shadecolor}{rgb}{.97, .97, .97}
\definecolor{messagecolor}{rgb}{0, 0, 0}
\definecolor{warningcolor}{rgb}{1, 0, 1}
\definecolor{errorcolor}{rgb}{1, 0, 0}
\newenvironment{knitrout}{}{} % an empty environment to be redefined in TeX

\usepackage{alltt}
\usepackage[fontsize=13pt]{scrextend}
\usepackage[utf8]{vietnam}

\makeatletter
\title{Ứng dụng, đánh giá, và so sánh một số mô hình phân loại vào việc phân loại khách hàng thẻ tín dụng}\let\Title\@title
\author{Nguyễn Đức Hiếu}\let\Author\@author
\makeatother
%%%%%%%%%%%%%%%%%%%%%%%%%
% TITLE PAGE FORMATTING %
%%%%%%%%%%%%%%%%%%%%%%%%%
\usepackage{afterpage}
\usepackage{xcolor}
\usepackage{graphicx}

%%%%%%%%%%%%%%%%%%%%%%%%
% PARAGRAPH FORMATTING %
%%%%%%%%%%%%%%%%%%%%%%%%
\usepackage{enumitem}
% Indent first paragraphs after each sections
\usepackage{indentfirst}

% Document formatting
\usepackage{mathptmx}
\renewcommand{\baselinestretch}{1.3}
\usepackage[a4paper]{geometry}
  \geometry{
  top=25mm,
  left=35mm,
  bottom=25mm,
  right=25mm
  }
%%%%%%%%%%%%%%%%%%%%%
% HEADER AND FOOTER %
%%%%%%%%%%%%%%%%%%%%%
\usepackage{etoolbox}
\patchcmd{\chapter}{\thispagestyle{plain}}{\thispagestyle{fancy}}{}{}
\usepackage{fancyhdr}
\pagestyle{fancy}
\fancyhf{}
\lhead{}
\chead{\normalsize Chuyên đề thực tập chuyên ngành Toán Kinh tế}
\rhead{}
\lfoot{}
\cfoot{\normalsize 11131371 - Nguyễn Đức Hiếu}
\rfoot{\normalsize Trang \thepage}
\renewcommand{\headrulewidth}{0.4pt}
\renewcommand{\footrulewidth}{0.4pt}
%%%%%%%%%%%%%%%%%%%%%%
% SECTION FORMATTING %
%%%%%%%%%%%%%%%%%%%%%%


\usepackage{titlesec}

\titleformat{\chapter}[display]
{\centering\Large\bfseries}
{\MakeUppercase{\chaptertitlename}  \thechapter}{1em}
{\MakeUppercase}

\titleformat{name=\chapter,numberless}[display]
  {\normalfont\Large\bfseries\filcenter}{}{1ex}
  {\MakeUppercase}[\vspace{1ex}]

\titleformat{\section}[hang]
{\bfseries}
{\thesection}{1em}
{\MakeUppercase}

\titleformat{\subsection}[hang]
{}
{\thesubsection}{1em}
{}


%%%%%%%%%%%%
% CITATION %
%%%%%%%%%%%%
% Setting up for citation styles
\usepackage{natbib}

%%%%%%%%%%%%%%%%%%%
% TOC FORMATTING  %
%%%%%%%%%%%%%%%%%%%

% Package hyperref should be loaded last, as it rewrite many commands.
% \usepackage{tocloft}
\usepackage[colorlinks=true, citecolor=blue, linkcolor=blue]{hyperref}
% Customise for table of contents labeling
\setcounter{tocdepth}{3}
\setcounter{secnumdepth}{4}

%%%%%%%%%%%%%%%%%%%%%%%%%%%%%%%%%%%%%%%%%%%%
%%%%%%%%%%%%%%%%%%%%%%%%%%%%%%%%%%%%%%%%%%%%
%%%%---------DOCUMENT CONTENTS----------%%%%
%%%%%%%%%%%%%%%%%%%%%%%%%%%%%%%%%%%%%%%%%%%%
%%%%%%%%%%%%%%%%%%%%%%%%%%%%%%%%%%%%%%%%%%%%
\IfFileExists{upquote.sty}{\usepackage{upquote}}{}
\begin{document}
\bibliographystyle{apalike}
%%%%%%%%%%%%%%%%%%%%%%%%%%%%%%%
%         TITLE PAGE          %
%%%%%%%%%%%%%%%%%%%%%%%%%%%%%%%
\begin{titlepage}

\definecolor{titlepagecolor}{RGB}{1, 81, 159 }
% \definecolor{titlepagecolor}{RGB}{128, 0, 0 }
\color{white}
\pagecolor{titlepagecolor}\afterpage{\nopagecolor}
\large
\centering
\textbf{TRƯỜNG ĐẠI HỌC KINH TẾ QUỐC DÂN}

\textbf{KHOA TOÁN KINH TẾ}
\vspace{5mm}

\includegraphics[width=0.4\textwidth]{./Cover/neu-logo.png}\par\vspace{1cm}
{\bfseries\scshape\Huge Chuyên đề thực tập}


\begin{description}[leftmargin=6cm,style=nextline]
\item[Chuyên ngành:] Toán Kinh tế
\item[Đề tài:] \Title
\item[Sinh viên thực hiện:] Nguyễn Đức Hiếu
\item[Mã sinh viên:] 11131371
\item[Lớp:] Toán Kinh tế 55
\item[Giảng viên hướng dẫn:] PGS. Nguyễn Thị Minh
\end{description}

\centering
\vfill
% Bottom of the page
\makebox[0pt]{\rule{0.5\textwidth}{1pt}}

{\large Hà Nội,  \today\par}
\end{titlepage}


%%%%%%%%%%%%%%%
% Lời mở đầu %
%%%%%%%%%%%%%%%
\chapter*{Lời mở đầu}
\addcontentsline{toc}{section}{Lời mở đầu}

% LỜI MỞ ĐẦU  
%
% - Nêu vấn đề
Đối với các ngân hàng việc chấm điểm tín dụng và phân loại các khách hàng là yếu tố thiết yếu cho lợi nhuận của ngân hàng.
%
Phương pháp truyền thống của việc ra quyết định có cho một cá nhân cụ thể vay hay không là dựa trên đánh giá cảm tính dựa trên kinh nghiệm cá nhân.
%
Tuy nhiên, sự phát triển về quy mô của nền kinh tế đã tạo ra sức ép về nhu cầu vay, đi kèm với đó là sự cạnh tranh giữa các ngân hàng và công nghệ máy tính ngày càng phát triển đã khiến cho việc sử dụng các mô hình thống kê trong việc phân loại các khách hàng tín dụng là bắt buộc đối với các ngân hàng trên thế giới mà ở Việt Nam cũng không phải là ngoại lệ.
%

Đã có rất nhiều nghiên cứu với các phương pháp khác nhau dành cho việc chấm điểm tín dụng.
Hiệu quả của các phương pháp khác nhau là có sự khác biệt, phụ thuộc vào việc ứng dụng các mô hình và trường hợp cụ thể của từng bộ số liệu.
Trong bài này, chúng ta sẽ tiếp cận đến một số phương pháp phân loại các khách hàng tín dụng và rút ra một số kết luận về việc sử dụng các phương pháp thống kê khác nhau sao cho hợp lý.

% - Hướng tiếp cận 
Văn bản này được soạn thảo bằng \LaTeX, \texttt{Sweave} và \texttt{knitr} 
\citep{r:knitr}. Tất cả phân tích được thực hiện trên phần mềm thống kê 
R version 3.3.3 (2017-03-06) \citep{r:rbase},  
các phân tích cụ thể được thực hiện sử dụng các gói mở rộng \texttt{ggplot2} \citep{r:ggplot2},...

%  Cảm ơn bla bla
Em xin cảm ơn giáo viên hướng dẫn, cô Nguyễn Thị Minh, trưởng khoa 
Toán Ứng dụng trong Kinh tế, cùng với các thầy cô giáo khác trong khoa đã tạo điều kiện cho em thực hiện đề tài này.

%%%%%%%
% TOC %
%%%%%%%
\clearpage\tableofcontents
\addcontentsline{toc}{section}{Mục lục}

\listoftables
\addcontentsline{toc}{section}{Danh sách bảng}

\listoffigures
\addcontentsline{toc}{section}{Danh sách hình}

%%%%%%%%%%%%
% Chương 1 %
%%%%%%%%%%%%
\chapter{Tổng quan}
\section{Khái niệm}

\section{Một số mô hình chấm điểm tín dụng}

\section{Thực trạng sử dụng các mô hình tín dụng tại Việt Nam}

\section{Kết luận}

%%%%%%%%%%%%
% Chương 2 %
%%%%%%%%%%%%

\chapter{Các phương pháp phân loại khách hàng sử dụng thẻ tín dụng}
\section{Các mô hình phân loại}
\subsection{Mô hình logistic}
\subsection{Mô hình phân loại tuyến tính}
\subsection{Mô hình SVM (Support Vector Machine)}

\section{Đánh giá mô hình}
\subsection{Đường ROC và phần diện tích dưới đường cong (AUC)}
\subsection{Thang đo H}

%%%%%%%%%%%%
% Chương 3 %
%%%%%%%%%%%%

\chapter{Tình huống nghiên cứu}


\section{Số liệu và các biến số}

Chúng ta thực hành trên bộ số liệu mẫu bao gồm 30000 quan sát và 25 biến bao gồm tình trạng trả nợ, các thông tin nhân khẩu học cơ bản cùng với số liệu về tín dụng và tình trạng hồ sơ của các khách hàng thẻ tín dụng ở Đài Loan từ tháng 4 năm 2005 đến tháng 9 năm 2005.

Các tên biến đã được thay đổi để tiện lợi cho việc đọc hiểu và phân tích, cụ thể như sau:



\begin{description}
  \item [\texttt{ID}] Số ID của mỗi khách hàng tín dụng
  \item [\texttt{LIMIT\_BAL}] Lượng tín dụng cho vay tính bằng Đô la Đài Loan (bao gồm cả các khoản vay cá nhân và các khoản vay với thẻ tín dụng phụ)
  \item [\texttt{SEX}] Giới tính (1=Nam, 2=Nữ)
  \item [\texttt{EDUCATION}] (1=sau đại học, 2=đại học, 3=phổ thông, 4=khác, 5=không rõ, 6=không rõ)
  \item [\texttt{MARRIAGE}] Trạng thái hôn nhân (1=đã cưới, 2=độc thân, 3=khác)
  \item [\texttt{AGE}] Số tuổi tính bằng năm
  \item [\texttt{PAY\_0}] Tình trạng hồ sơ vào thời điểm tháng 9/2005 (-1=trả đúng hạn, 1=chậm 1 tháng, 2=chậm 2 tháng, ... 8=chậm 8 tháng, 9=chậm 9 tháng hoặc nhiều hơn)
  \item [\texttt{PAY\_2}] Tình trạng hồ sơ vào thời điểm tháng 8/2005 (thang điểm như trên)
  \item [\texttt{PAY\_3}] Tình trạng hồ sơ vào thời điểm tháng 7/2005 (thang điểm như trên)
  \item [\texttt{PAY\_4}] Tình trạng hồ sơ vào thời điểm tháng 6/2005 (thang điểm như trên)
  \item [\texttt{PAY\_5}] Tình trạng hồ sơ vào thời điểm tháng 5/2005 (thang điểm như trên)
  \item [\texttt{PAY\_6}] Tình trạng hồ sơ vào thời điểm tháng 4/2005 (thang điểm như trên)
  \item [\texttt{BILL\_AMT1}] Hóa đơn thanh toán vào thời điểm 9/2005 (Đô la Đài Loan)
  \item [\texttt{BILL\_AMT2}] Hóa đơn thanh toán vào thời điểm 8/2005 (Đô la Đài Loan)
  \item [\texttt{BILL\_AMT3}] Hóa đơn thanh toán vào thời điểm 7/2005 (Đô la Đài Loan)
  \item [\texttt{BILL\_AMT4}] Hóa đơn thanh toán vào thời điểm 6/2005 (Đô la Đài Loan)
  \item [\texttt{BILL\_AMT5}] Hóa đơn thanh toán vào thời điểm 5/2005 (Đô la Đài Loan)
  \item [\texttt{BILL\_AMT6}] Hóa đơn thanh toán vào thời điểm 4/2005 (Đô la Đài Loan)
  \item [\texttt{PAY\_AMT1}] Lượng tiền đã thanh toán vào thời điểm tháng 9/2015 (Đô la Đài Loan)
  \item [\texttt{PAY\_AMT2}] Lượng tiền đã thanh toán vào thời điểm tháng 8/2015 (Đô la Đài Loan)
  \item [\texttt{PAY\_AMT3}] Lượng tiền đã thanh toán vào thời điểm tháng 7/2015 (Đô la Đài Loan)
  \item [\texttt{PAY\_AMT4}] Lượng tiền đã thanh toán vào thời điểm tháng 6/2015 (Đô la Đài Loan)
  \item [\texttt{PAY\_AMT5}] Lượng tiền đã thanh toán vào thời điểm tháng 5/2015 (Đô la Đài Loan)
  \item [\texttt{PAY\_AMT6}] Lượng tiền đã thanh toán vào thời điểm tháng 4/2015 (Đô la Đài Loan)
  \item [\texttt{default}] Có trả nợ hay không (1=có, 0=không)
\end{description}

\section{Ứng dụng mô hình logit}

\section{Ứng dụng mô hình phân loại tuyến tính}

\section{Ứng dụng mô hình SVM}

%%%%%%%%%%%%
% Kết luận %
%%%%%%%%%%%%

\chapter{Kết luận}

%%%%%%%%%%%%%%%%%%%%%%%
% Danh mục tham khảo  %
%%%%%%%%%%%%%%%%%%%%%%%
\bibliography{./reference.bib}
\addcontentsline{toc}{section}{Tài liệu tham khảo}

\end{document}
