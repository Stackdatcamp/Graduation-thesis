\documentclass[a4paper]{report}\usepackage[]{graphicx}\usepackage[]{color}
%% maxwidth is the original width if it is less than linewidth
%% otherwise use linewidth (to make sure the graphics do not exceed the margin)
\makeatletter
\def\maxwidth{ %
  \ifdim\Gin@nat@width>\linewidth
    \linewidth
  \else
    \Gin@nat@width
  \fi
}
\makeatother

\definecolor{fgcolor}{rgb}{0.345, 0.345, 0.345}
\newcommand{\hlnum}[1]{\textcolor[rgb]{0.686,0.059,0.569}{#1}}%
\newcommand{\hlstr}[1]{\textcolor[rgb]{0.192,0.494,0.8}{#1}}%
\newcommand{\hlcom}[1]{\textcolor[rgb]{0.678,0.584,0.686}{\textit{#1}}}%
\newcommand{\hlopt}[1]{\textcolor[rgb]{0,0,0}{#1}}%
\newcommand{\hlstd}[1]{\textcolor[rgb]{0.345,0.345,0.345}{#1}}%
\newcommand{\hlkwa}[1]{\textcolor[rgb]{0.161,0.373,0.58}{\textbf{#1}}}%
\newcommand{\hlkwb}[1]{\textcolor[rgb]{0.69,0.353,0.396}{#1}}%
\newcommand{\hlkwc}[1]{\textcolor[rgb]{0.333,0.667,0.333}{#1}}%
\newcommand{\hlkwd}[1]{\textcolor[rgb]{0.737,0.353,0.396}{\textbf{#1}}}%
\let\hlipl\hlkwb

\usepackage{framed}
\makeatletter
\newenvironment{kframe}{%
 \def\at@end@of@kframe{}%
 \ifinner\ifhmode%
  \def\at@end@of@kframe{\end{minipage}}%
  \begin{minipage}{\columnwidth}%
 \fi\fi%
 \def\FrameCommand##1{\hskip\@totalleftmargin \hskip-\fboxsep
 \colorbox{shadecolor}{##1}\hskip-\fboxsep
     % There is no \\@totalrightmargin, so:
     \hskip-\linewidth \hskip-\@totalleftmargin \hskip\columnwidth}%
 \MakeFramed {\advance\hsize-\width
   \@totalleftmargin\z@ \linewidth\hsize
   \@setminipage}}%
 {\par\unskip\endMakeFramed%
 \at@end@of@kframe}
\makeatother

\definecolor{shadecolor}{rgb}{.97, .97, .97}
\definecolor{messagecolor}{rgb}{0, 0, 0}
\definecolor{warningcolor}{rgb}{1, 0, 1}
\definecolor{errorcolor}{rgb}{1, 0, 0}
\newenvironment{knitrout}{}{} % an empty environment to be redefined in TeX

\usepackage{amsmath}
\usepackage{alltt}
\usepackage[fontsize=13pt]{scrextend}
\usepackage[utf8]{vietnam}
\usepackage{verbatim} 
\usepackage{lipsum}
\usepackage{listings}



\makeatletter
\title{Ứng dụng, đánh giá, và so sánh một số mô hình phân loại trong phân loại khách hàng thẻ tín dụng}\let\Title\@title
\author{Nguyễn Đức Hiếu}\let\Author\@author
\makeatother
%%%%%%%%%%%%%%%%%%%%%%%%%
% TITLE PAGE FORMATTING %
%%%%%%%%%%%%%%%%%%%%%%%%%
\usepackage{afterpage}
\usepackage{xcolor}
\usepackage{graphicx}

%%%%%%%%%%%%%%%%%%%%%%%%
% PARAGRAPH FORMATTING %
%%%%%%%%%%%%%%%%%%%%%%%%
\usepackage{enumitem}
% Indent first paragraphs after each sections
\usepackage{indentfirst}

% Document formatting
\usepackage{mathptmx}
\renewcommand{\baselinestretch}{1.3}
\usepackage[a4paper]{geometry}
  \geometry{
  top=25mm,
  left=35mm,
  bottom=25mm,
  right=25mm
  }
%%%%%%%%%%%%%%%%%%%%%
% HEADER AND FOOTER %
%%%%%%%%%%%%%%%%%%%%%
\usepackage{etoolbox}
\patchcmd{\chapter}{\thispagestyle{plain}}{\thispagestyle{fancy}}{}{}
\usepackage{fancyhdr}
\pagestyle{fancy}
\fancyhf{}
\lhead{}
\chead{\normalsize Chuyên đề thực tập chuyên ngành Toán Kinh tế}
\rhead{}
\lfoot{}
\cfoot{\normalsize 11131371 - Nguyễn Đức Hiếu}
\rfoot{\normalsize Trang \thepage}
\renewcommand{\headrulewidth}{0.4pt}
\renewcommand{\footrulewidth}{0.4pt}
%%%%%%%%%%%%%%%%%%%%%%
% SECTION FORMATTING %
%%%%%%%%%%%%%%%%%%%%%%


\usepackage{titlesec}

\titleformat{\chapter}[display]
  {\centering\Large\bfseries}
  {\MakeUppercase{\chaptertitlename}  \thechapter}{1em}
  {\MakeUppercase}

\titleformat{name=\chapter,numberless}[display]
  {\normalfont\Large\bfseries\filcenter}{}{1ex}
  {\MakeUppercase}[\vspace{1ex}]

\titleformat{\section}[hang]
  {\bfseries}
  {\thesection}{1em}
  {\MakeUppercase}

\titleformat{\subsection}[hang]
  {\bfseries}
  {\thesubsection}{1em}
  {}

% Appearance on table of contents
\usepackage{titletoc}
\titlecontents{chapter}% <section-type>
  [0pt]% <left>
  {\bfseries}% <above-code>
  {\color{blue}CHƯƠNG \thecontentslabel:\quad}% <numbered-entry-format>
  {}% <numberless-entry-format>
  {\titlerule*[1pc]{.}\contentspage}% <filler-page-format>


\renewcommand{\thechapter}{\Roman{chapter}}
\renewcommand{\thesection}{\arabic{section}}

%%%%%%%%%%%%%%%%%%%%%%  
% FIGURES AND TABLES %
%%%%%%%%%%%%%%%%%%%%%%
\usepackage[section]{placeins}
\usepackage[justification=centering]{caption}

%%%%%%%%%%%%%%%%%%%
% TOC FORMATTING  %
%%%%%%%%%%%%%%%%%%%
% \usepackage{tocloft}
% \renewcommand{\cftchappresnum}{\MakeUppercase\chaptername\ }
% % \renewcommand{\cftchapaftersnum}{          }
% % \renewcommand{\cftchapaftersnumb}{\newline}
% \renewcommand{\cftchapdotsep}{\cftdotsep}
% Package hyperref should be loaded last, as it rewrites many commands.
% \usepackage{tocloft}
\usepackage[colorlinks=true, citecolor=blue, linkcolor=blue]{hyperref}
\usepackage[all]{hypcap}
% Customise for table of contents labeling
\setcounter{tocdepth}{3}
\setcounter{secnumdepth}{4}
%%%%%%%%%%%%
% CITATION %
%%%%%%%%%%%%
% Setting up for citation styles
% \usepackage{natbib}
\usepackage[backend=bibtex,bibstyle=luanan,sorting=ydnt,style=authoryear]{biblatex}
\addbibresource{reference.bib}
% Add number to bibliography
\defbibenvironment{bibliography}
  {\enumerate
     {}
     {\setlength{\leftmargin}{\bibhang}%
      \setlength{\itemindent}{-\leftmargin}%
      \setlength{\itemsep}{\bibitemsep}%
      \setlength{\parsep}{\bibparsep}}}
  {\endenumerate}
  {\item}
%%%%%%%%%%%%%%%%%%%%%%%%%%%%%%%%%%%%%%%%%%%%
%%%%%%%%%%%%%%%%%%%%%%%%%%%%%%%%%%%%%%%%%%%%
%%%%---------DOCUMENT CONTENTS----------%%%%
%%%%%%%%%%%%%%%%%%%%%%%%%%%%%%%%%%%%%%%%%%%%
%%%%%%%%%%%%%%%%%%%%%%%%%%%%%%%%%%%%%%%%%%%%
\IfFileExists{upquote.sty}{\usepackage{upquote}}{}
\begin{document}
%%%%%%%%%%%%%%%%%%%%%%%%%%%%%%%
%         TITLE PAGE          %
%%%%%%%%%%%%%%%%%%%%%%%%%%%%%%%
\begin{titlepage}

\definecolor{titlepagecolor}{RGB}{1, 81, 159 }
% \definecolor{titlepagecolor}{RGB}{128, 0, 0 }
\color{white}
\pagecolor{titlepagecolor}\afterpage{\nopagecolor}
\large
\centering
\textbf{TRƯỜNG ĐẠI HỌC KINH TẾ QUỐC DÂN}

\textbf{KHOA TOÁN KINH TẾ}
\vspace{5mm}

\includegraphics[width=0.4\textwidth]{./Cover/neu-logo.png}\par\vspace{1cm}
{\bfseries\scshape\Huge Chuyên đề thực tập}


\begin{description}[leftmargin=6cm,style=nextline]
\item[Chuyên ngành:] Toán Kinh tế
\item[Đề tài:] \Title
\item[Sinh viên thực hiện:] Nguyễn Đức Hiếu
\item[Mã sinh viên:] 11131371
\item[Lớp:] Toán Kinh tế 55
\item[Giảng viên hướng dẫn:] PGS. Nguyễn Thị Minh
\end{description}

\centering
\vfill
% Bottom of the page
\makebox[0pt]{\rule{0.5\textwidth}{1pt}}

{\large Hà Nội,  \today\par}
\end{titlepage}


%%%%%%%
% TOC %
%%%%%%%
\clearpage\tableofcontents
\addcontentsline{toc}{chapter}{Mục lục}

\listoftables
\addcontentsline{toc}{chapter}{Danh sách bảng}

\listoffigures
\addcontentsline{toc}{chapter}{Danh sách hình}

%%%%%%%%%%%%%%%
% Lời mở đầu %
%%%%%%%%%%%%%%%
\chapter*{Lời mở đầu}
\addcontentsline{toc}{chapter}{Lời mở đầu}

% LỜI MỞ ĐẦU  
%
% - Nêu vấn đề
Đối với các ngân hàng việc chấm điểm tín dụng và phân loại các khách hàng là yếu tố thiết yếu cho lợi nhuận của ngân hàng.
%
Phương pháp truyền thống của việc ra quyết định có cho một cá nhân cụ thể vay hay không là dựa trên đánh giá cảm tính dựa trên kinh nghiệm cá nhân.
%
Tuy nhiên, sự phát triển về quy mô của nền kinh tế đã tạo ra sức ép về nhu cầu vay, đi kèm với đó là sự cạnh tranh giữa các ngân hàng và công nghệ máy tính ngày càng phát triển đã khiến cho việc sử dụng các mô hình thống kê trong việc phân loại các khách hàng tín dụng là bắt buộc đối với các ngân hàng trên thế giới mà ở Việt Nam cũng không phải là ngoại lệ.
%

Vậy, phương pháp ước lượng nào có thể giúp chúng ta xây dựng được hệ thống chấm điểm tín dụng chính xác nhất? Đã có một số nghiên cứu mang tính chất so sánh hiệu năng giữa các mô hình \parencite{baesens2003benchmarking, xiao2006comparative, lessmann2015benchmarking}. 
Sự khác biệt về hiệu năng của các phương pháp khác nhau là có, tuy nhiên hầu như là không đáng kể, và không phải các mô hình hiệu quả hơn đều là các mô hình mới và tân tiến.
Theo \textcite{thomas2010consumer}, cách hiệu quả để xây dựng một hệ thống lượng định hiệu quả là phối hợp nhiều mô hình khác nhau thay vì tìm kiếm một mô hình toàn diện có thể áp dụng với tất cả các ngân hàng.

Trong bài này, chúng ta sẽ tiếp cận đến một số phương pháp phân loại các khách hàng tín dụng phổ biến hiện nay và rút ra một số kết luận về việc sử dụng các phương pháp khác nhau sao cho hợp lý. Bài viết này được bố cục như sau:

\begin{itemize}
\item \textbf{Chương 1} đưa ra một cái nhìn tổng quan về lĩnh vực quản trị rủi ro tín dụng trong ngân hàng và đưa ra một số vấn đề của việc chấm điểm tín dụng tại các ngân hàng Việt Nam.
\item Các mô hình được thực hiện trong bài này sẽ được giới thiệu ở \textbf{Chương 2}, đi kèm với đó là một số chỉ tiêu sẽ được dùng để đánh giá mô hình trong bài này.
\item Trong \textbf{Chương 3}, chúng ta sẽ ứng dụng các phương pháp được giới thiệu ở \textbf{Chương 2} trong một bộ số liệu mẫu về các khách hàng thẻ tín dụng trong một ngân hàng ở Đài Loan.
\item Kết quả của các mô hình sẽ được thảo luận ở \textbf{Chương 4}, cùng với một số kết luận rút ra được sau khi áp dụng mô hình.
\end{itemize}

% - Hướng tiếp cận 
Đề tài này được soạn thảo bằng \LaTeX{} kết hợp với \texttt{Sweave} và \texttt{knitr} 
\parencite{r:knitr}. Tất cả phân tích được thực hiện trên phần mềm thống kê 
R version 3.4.0 (2017-04-21) \parencite{r:rbase},  
các phân tích cụ thể được thực hiện sử dụng các gói mở rộng \texttt{caret}\parencite{r:caret}, \texttt{tidyverse} \parencite{r:tidyverse}... Mô hình logit được thực hiện với gói \texttt{glmnet}\parencite{r:glmnet}. 
Mô hình SVM được thực hiện với gói \texttt{kernlab}\parencite{r:kernlab}, là một giao diện của phần mềm \texttt{LIBSVM} \parencite{CC01a} trong môi trường \texttt{R}.

%  Cảm ơn bla bla
Em xin cảm ơn giáo viên hướng dẫn, cô Nguyễn Thị Minh, cùng với các thầy cô giáo khác trong khoa đã tạo điều kiện cho em thực hiện đề tài này.

%%%%%%%%%%%%
% Chương 1 %
%%%%%%%%%%%%
\chapter{Tổng quan về quản trị rủi ro tín dụng đối với khách hàng cá nhân}

\section{Một số khái niệm}

\section{Thực trạng của việc chấm điểm tín dụng tại Việt Nam}

\section{Kết luận}

%%%%%%%%%%%%
% Chương 2 %
%%%%%%%%%%%%

\chapter{Các phương pháp phân loại khách hàng vay thẻ tín dụng}
\section{Các mô hình phân loại}

\subsection{Mô hình logit}
% knit_child('./Ch2/2.1/def-logistics.rnw')

\subsection{Mô hình phân loại tuyến tính}
\subsection{Mô hình SVM (Support Vector Machine)}

\section{Đánh giá mô hình}
\subsection{Đường ROC và phần diện tích dưới đường cong (AUC)}
\subsection{Thang đo H}

%%%%%%%%%%%%
% Chương 3 %
%%%%%%%%%%%%

\chapter{Tình huống nghiên cứu}

\section{Số liệu và các biến số}
% knit_child('./Ch3/3.1/info.rnw')


\section{Ứng dụng mô hình logit}
% knit_child('./Ch3/3.2/var_select.rnw')

\section{Ứng dụng mô hình phân loại tuyến tính}

\section{Ứng dụng mô hình SVM}

%%%%%%%%%%%%
% Kết luận %
%%%%%%%%%%%%

\chapter{Kết luận}

%%%%%%%%%%%
% Phụ lục %
%%%%%%%%%%%
\appendix
\titlecontents{chapter}[0pt]{\bfseries}
  {\color{blue}PHỤ LỤC \thecontentslabel:\quad}
  {}{\titlerule*[1pc]{.}\contentspage}

\chapter{Thông tin về phiên làm việc trên R}

\begin{knitrout}\small
\definecolor{shadecolor}{rgb}{0.969, 0.969, 0.969}\color{fgcolor}\begin{kframe}
\begin{verbatim}
## R version 3.4.0 (2017-04-21)
## Platform: x86_64-pc-linux-gnu (64-bit)
## Running under: Ubuntu 16.04.2 LTS
## 
## Matrix products: default
## BLAS: /usr/lib/libblas/libblas.so.3.6.0
## LAPACK: /usr/lib/lapack/liblapack.so.3.6.0
## 
## locale:
##  [1] LC_CTYPE=en_US.UTF-8    LC_NUMERIC=C           
##  [3] LC_TIME=vi_VN           LC_COLLATE=en_US.UTF-8 
##  [5] LC_MONETARY=vi_VN       LC_MESSAGES=en_US.UTF-8
##  [7] LC_PAPER=vi_VN          LC_NAME=C              
##  [9] LC_ADDRESS=C            LC_TELEPHONE=C         
## [11] LC_MEASUREMENT=vi_VN    LC_IDENTIFICATION=C    
## 
## attached base packages:
## [1] stats     graphics  grDevices utils     datasets 
## [6] methods   base     
## 
## other attached packages:
##  [1] glmnet_2.0-5    foreach_1.4.3   Matrix_1.2-8   
##  [4] ggfortify_0.4.1 GGally_1.3.0    caret_6.0-76   
##  [7] ggplot2_2.2.1   lattice_0.20-35 broom_0.4.2    
## [10] dplyr_0.5.0     tidyr_0.6.1     readr_1.1.0    
## [13] knitr_1.15.1   
## 
## loaded via a namespace (and not attached):
##  [1] Rcpp_0.12.10       RColorBrewer_1.1-2
##  [3] nloptr_1.0.4       compiler_3.4.0    
##  [5] plyr_1.8.4         iterators_1.0.8   
##  [7] tools_3.4.0        lme4_1.1-13       
##  [9] evaluate_0.10      tibble_1.3.0      
## [11] gtable_0.2.0       nlme_3.1-131      
## [13] mgcv_1.8-17        psych_1.7.3.21    
## [15] DBI_0.6-1          parallel_3.4.0    
## [17] SparseM_1.77       gridExtra_2.2.1   
## [19] stringr_1.2.0      MatrixModels_0.4-1
## [21] hms_0.3            stats4_3.4.0      
## [23] grid_3.4.0         nnet_7.3-12       
## [25] reshape_0.8.6      R6_2.2.0          
## [27] foreign_0.8-67     minqa_1.2.4       
## [29] reshape2_1.4.2     car_2.1-4         
## [31] magrittr_1.5       splines_3.4.0     
## [33] scales_0.4.1       codetools_0.2-15  
## [35] ModelMetrics_1.1.0 MASS_7.3-47       
## [37] assertthat_0.2.0   pbkrtest_0.4-7    
## [39] mnormt_1.5-5       colorspace_1.3-2  
## [41] quantreg_5.33      stringi_1.1.5     
## [43] lazyeval_0.2.0     munsell_0.4.3
\end{verbatim}
\end{kframe}
\end{knitrout}

% \chapter{Code}




%%%%%%%%%%%%%%%%%%%%%%%
% Danh mục tham khảo  %
%%%%%%%%%%%%%%%%%%%%%%%
\printbibliography
\addcontentsline{toc}{chapter}{Tài liệu tham khảo}

\end{document}
